\input{latexKonf.tex}

\title{Dokumentacja}
\author{Mateusz banaszek \and Szymon Kozakiewicz \and Joanna Świętosławska}
\begin{document}
\maketitle
\section{Opis programu}
Program ma za zadanie udostępniać interfejs do komunikacji z full nodem sieci bitcoin. Pozawala na nawiązanie z nim połączenie oraz podstawową wymianę danych. Można więc uzyskać informacje o adresach innych węzłów czy otrzymać informacje o blokach i ich zawartości. Udostępnia też wysyłanie wiadomości czysto serwisowych takich jak ping, version czy verack. Możliwe są też różne sposoby na nawiązanie połączenia czyli UDP i TCP. 

\subsection{Realizowane funkcjonalności}

\begin{itemize}


\item Ustanowienie połączenia
\begin{itemize}
\item TCP
\item UDP
\end{itemize}

\item Komunikacja z użytkownikiem za pomocą konsoli
\item Znajdywanie adresów ip węzłów za pomocą DNS seed
\item Wysyłanie wiadomości:
\begin{itemize}
\item getaddr
\item addr
\item version
\item verack
\item ping
\item inv
\item getdata
\item getblocks
\end{itemize} 
\item Odbiór wiadomości
\begin{itemize}
\item addr
\item version
\item verack
\item ping
\item inv
\item tx

\end{itemize}
\end{itemize}

\subsection{Uruchomienie programu}
By uruchomić program należy wywołać polecenie $$python3 Console.py$$
Aplikacja była testowana dla pythona w wersji 3.6 na innych wersjach może on nie działać poprawnie.
\subsection{Opis działania}

Po uruchomieniu programu do konsoli można wpisać następujące polecenia:
\begin{itemize}
\item \textit{ping}:* wysyła wiadomość ping do wybranego hosta.
\item \textit{polacz}:* ustanawia połączenie z wybranym węzłem. Jeśli wcześniej żaden adres ip nie był ustawiony użytkownik zostanie poproszony o jego podanie. Wykonanie polecenia skutkuje wysłaniem do docelowego węzła wiadomości version. Następnie odebrana zostaje zwrotna wiadomość version od węzła. Na koniec następuje wymiana wiadomościami verack. Jeżeli nie uda się nawiązać połączenia TCP w przeciągu 5 sekund to próba połączenia kończy się niepowodzeniem.
\item \textit{help}:* Wyświetla możliwe do wpisania polecenia
\item \textit{ustaw adres}:* ustawia adres docelowego węzła
\item \textit{dns}:* wyszukiwanie adresów węzłów za pomocą dns. Wyszukane węzły zostają wypisane.
\item \textit{getaddr}: uzyskanie adresów innych węzłów z wybranego węzła
\item \textit{addr}: wysyła wiadomość addr do wybranego węzła
\item \textit{getblocks}: wyświetla bloki wybranego węzła(i hashe)
\item \textit{getdata}:  wyświetla dane z wybranego bloku. Użytkownik musi wcześniej podać hash identyfikujący wybrany blok.
\item \textit{inv}: Wysyła wiadomość inv
\end{itemize}

Uwaga! Tylko polecenia oznaczone gwiazdką da się wykonać bez wcześniejszego nawiązania połączenia z węzłem(za pomocą polecenia \textit{polacz})

\section{Implementacja}
Aplikacja została stworzona przy użyciu języka \textit{python} w wersji 3.6. Nie były używane żadne zewnętrzne biblioteki.


\end{document}